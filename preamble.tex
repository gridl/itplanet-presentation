%%% Преамбула

%%% Темы оформления
% https://www.hartwork.org/beamer-theme-matrix/
% \usetheme{Berkeley}
% \usetheme{Bergen}
% \usetheme{Szeged}

%%% Цветовые схемы
% \usecolortheme{beaver}
% \useinnertheme{circles}
% \useinnertheme{rectangles}
% \usecolortheme{crane}

\usetheme{SevGU} % Тема СевГУ

%%% Работа с русским языком
\usepackage{fontspec}
\usepackage{xunicode}
\usepackage{xltxtra}

\defaultfontfeatures{Ligatures=TeX}
\setmainfont{PT Sans} % http://fonts.ru/public/
% \setmonofont{PT Mono}

%%% Beamer по-русски
\newtheorem{rtheorem}{Теорема}
\newtheorem{rproof}{Доказательство}
\newtheorem{rexample}{Пример}

%%% Дополнительная работа с математикой
\usepackage{amsmath,amsfonts,amssymb,amsthm,mathtools} % AMS
\usepackage{icomma} % "Умная" запятая: $0,2$ --- число, $0, 2$ --- перечисление

%%% Номера формул
% \mathtoolsset{showonlyrefs=true} % Показывать номера только у тех формул, на которые есть \eqref{} в тексте.
% \usepackage{leqno} % Нумерация формул слева

%%% Свои команды
\DeclareMathOperator{\sgn}{\mathop{sgn}}

%%% Перенос знаков в формулах (по Львовскому)
\newcommand*{\hm}[1]{#1\nobreak\discretionary{}
{\hbox{$\mathsurround=0pt #1$}}{}}

%%% Работа с картинками
\usepackage{graphicx}    % Для вставки рисунков
\graphicspath{{images/}} % Каталоги с картинками
\setlength\fboxsep{3pt}  % Отступ рамки \fbox{} от рисунка
\setlength\fboxrule{1pt} % Толщина линий рамки \fbox{}
\usepackage{wrapfig}     % Обтекание рисунков текстом

%%% Работа с таблицами
\usepackage{array,tabularx,tabulary,booktabs} % Дополнительная работа с таблицами
\usepackage{longtable}  % Длинные таблицы
\usepackage{multirow}   % Слияние строк в таблице

%%% Другие пакеты
\usepackage{lastpage} % Узнать, сколько всего страниц в документе
\usepackage{soulutf8} % Модификаторы начертания
\usepackage{csquotes} % Еще инструменты для ссылок
% \usepackage[style=authoryear,maxcitenames=2,backend=biber,sorting=nty]{biblatex}
\usepackage{multicol} % Несколько колонок
\usepackage{etoolbox} % Логические операторы

%%% Картинки
\usepackage{tikz}
\usepackage{pgfplots,pgfplotstable}
\usepackage{verbatim,calc,ifthen}
